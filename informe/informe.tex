\documentclass[final,inline,a4paper,narroweqnarray]{ieee}
% In order to use the figure-defining commands in ieeefig.sty...
\usepackage{ieeefig}
% To use utf8 encoding
\usepackage[T1]{fontenc}
\usepackage[utf8]{inputenc}
\usepackage[spanish]{babel}
% to place figures
\usepackage{float}
\usepackage{graphicx}
% dont use geometry package
%\usepackage{geometry}
\begin{document}

%----------------------------------------------------------------------
% Title Information, Abstract and Keywords
%----------------------------------------------------------------------
\title[TP N$^o$ 1: Wiretapping]{%
Trabajo Práctico N$^o$ 1: Wiretapping}

% format author this way for journal articles.
\author[Barrios, Benegas, Caravario, Rodriguez]{%
	Leandro Ezequiel Barrios,
	\and
	Gonzalo Benegas,
	\and
	Martin Caravario, 
	\and
	Pedro Rodriguez 
}

% make the title
\maketitle

% do the abstract
\begin{abstract}

En el presente Trabajo Práctico utilizaremos algunas de las técnicas
provistas por la teoría de la información para estudiar y analizar
algunas redes de información. El objetivo será distinguir diversos
aspectos de la red de manera analítica. Para cumplir con nuestro
objetivo, haremos uso de dos herramientas modernas de manipulación y
análisis de paquetes: Wireshark y Scapy.

\end{abstract}

% start the main text ...

%----------------------------------------------------------------------
% SECTION I: Introduction
%----------------------------------------------------------------------
\section{ Introducción }

Construimos una herramienta que hace uso de la función
\texttt{``sniff''}, provista por la librería \textbf{Scapy} de
Python. Esta nos permitió activar el \texttt{modo promiscuo}, o
\texttt{monitor} en el caso de las placas wireless. Esto nos permitió
escuchar la red durante cierto tiempo, obteniendo todos los paquetes
que llegaban a nuestra placa de red. A partir de estos datos que
guardamos en un archivo \texttt{pcap}, se definieron dos fuentes de
información, con las cuales fuimos capaces de encontrar nodos y
protocolos distinguidos en la red. Para su visualización, elegimos la
realización de gráficos de torta e histogramas, ya que los
consideramos los más apropiados.

Para cada una de las mediciones consideramos las siguientes fuentes: 
\begin{itemize}

  \item $S = \{s_{1} \dots s_{n}\}$, provista por la cátedra, donde
  $s_{i}$ es el valor del campo \emph{type} de cada paquete de capa
  2.

  \item $S_{1} = \{s_{1} \dots s_{n}\} $, determinada por nosotros,
  donde $s_i$ es el valor del campo destino (MAC) cada paquete de
  capa 2 de tipo ARP.

\end{itemize}

\medskip

Para entender qué es lo que se obtendrá al efectuar estas mediciones,
hay que aclarar que \texttt{ARP} es un protocolo de la capa de enlace
de datos, responsable de encontrar la dirección de capa 2
(\texttt{Ethernet MAC}) que corresponde a una determinada dirección IP
(dirección de capa 3 de red).

Es decir, cada vez que un host quiere comunicarse con otro y su
dirección \texttt{MAC} no se encuentra dentro de su tabla
\texttt{ARP}, debe enviar un paquete who-has broadcast para determinar
la dirección \texttt{MAC} del host destino. De este modo, todos los
hosts del dominio de colisión de la máquina en la que se efectúa la
medición reciben dicho paquete, siendo respondido el mismo únicamente
por el host requerido, mediante un paquete \emph{reply}.

Para distinguir \emph{nodos} (símbolos) en este contexto, tomamos a
aquellos cuya probabilidad de aparición era alta, de forma tal que la
información provista por el mismo fuera menor a la entropía de la
fuente a la cuál el símbolo pertenecía. 

Tomamos esta decisión porque, según Shannon, el nivel de entropía de
una fuente habla de la máxima compresibilidad de cada bit en un
mensaje enviado con una codificación óptima ($H(F) \leq L(C)$). Luego,
si la fuente presenta símbolos cuya cantidad de información está por
debajo de la entropía, son símbolos que tienen mucha probabilidad de
aparecer en un mensaje en comparación con los otros, y podría ser
conveniente representar a estos símbolos con menos bits que al resto,
para así disminuir la longitud media del código. Por ejemplo,
supongamos que enviamos números binarios, y sabemos que el símbolo $X
= "00000000"$ aparece en los mensajes la mitad del tiempo y que el
resto de las tiras pueden aparecer todas con la misma probabilidad.
Entonces, $X$ brindaría una cantidad de información por debajo de la
entropía, y convendría representar a esa tira simplemente con un $0$,
y al resto de las tiras prefijarlas con el $1$. De esta manera, se
ahorraría en promedio $7$.$1/2+(-1)$.$1/2 = 3$ bits en cada mensaje.

Entonces, también analizaremos las entropías de las fuentes en cada
una de las escuchas de red realizadas, y trataremos de concluir cuál
de las dos fuentes elegidas es \emph{más compresible}.

%----------------------------------------------------------------------
% SECTION II: Desarrollo - Fuente S
%----------------------------------------------------------------------
\section{Desarrollo - Fuente S}
  %--------------------------------------------------------------------
  % SUBSECTION II-A: CASA
  %--------------------------------------------------------------------
  \subsection{Casa}

  La siguiente medición fue realizada en una red LAN hogareña, en el
  horario de las 15:00 hs, por un lapso de 3 hs. La herramienta
  utilizada fue la indicada en el enunciado del trabajo. A esta LAN
  hubo conectadas 5 computadoras, una impresora y 3 celulares al
  momento de la medición. Los resultados obtenidos fueron los
  siguientes

    %------------------------------------------------------------------
    % TORTA CASA
    %------------------------------------------------------------------

    \begin{figure}[ht]\begin{center}
      \includegraphics[width=0.5\textwidth]{%
      ../output/marto-casa-3hs-s-pie.png}
      \vspace{-3em}
      \caption{S: Casa - Torta}
      \label{marto-casa-3hs-s-pie}
    \end{center}\end{figure}

    %------------------------------------------------------------------
    % HISTOGRAMA CASA
    %------------------------------------------------------------------
    
    \begin{figure}[ht]\begin{center}
     \includegraphics[width=0.5\textwidth]{%
      ../output/marto-casa-3hs-s-histogram.png}
      \caption{S: Casa - Histograma}
      \label{marto-casa-3hs-s-histogram}
    \end{center}\end{figure}

    El protocolo que más fue escuchado fue \texttt{IPv4}, tal como se
    observa en la figura \ref{marto-casa-3hs-s-pie}, la cantidad de
    paquetes  de este tipo fue del 98\% mientras que la de paquetes
    \texttt{ARP} fue del 1\%. Esto permite concluir que en el caso de
    la red hogareña, el overhead del tipo \texttt{ARP} en un tiempo de
    3 horas es practicamente nulo, con respecto al total de paquetes
    de la red.

    La entropía de la fuente propuesta es de 0.095, lo que indica que
    los símbolos emitidos por la fuente son muy previsibles. Esto se
    puede observar en la figura \ref{marto-casa-3hs-s-histogram},
    donde deja por debajo de ella al protocolo \texttt{IPv4} que, al
    presentar una mayor probabilidad de aparición en la fuente,
    provoca que la información que aporte sea poca y lo destaque como
    nodo distinguido. A diferencia de
    \texttt{IPv4}, podemos encontrar al protocolo \texttt{802.1X} que al
    tener poca probabilidad de aparición $(2.35 * 10^-5)$, aporta mucha
    información, siendo el protocolo que más aporta.
      
    
    %------------------------------------------------------------------
    % SUBSECTION II-B: TECHINT
    %------------------------------------------------------------------
    \subsection{Techint}

    Esta medición fue realizada en la empresa Techint, en el horario
    de las 11:00 am, por un lapso de 30 minutos. Se utilizó la
    herramienta desarrollada en el ejercicio anterior. Se desconoce la
    cantidad de computadoras o la topología de la red medida. La
    conexión a la red fue efectuada mediante un cable de ethernet. Los
    resultados fueron los siguientes.
    
    %------------------------------------------------------------------
    % TORTA TECHINT
    %------------------------------------------------------------------
    
    \begin{figure}[ht]\begin{center}
      \includegraphics[width=0.5\textwidth]{%
      ../output/techint-s-pie.png}
      \vspace{-3em}
      \caption{S: Techint - Torta}
      \label{techint-s-pie}
    \end{center}\end{figure}
 
    %------------------------------------------------------------------
    % HISTOGRAMA TECHINT
    %------------------------------------------------------------------
     
    \begin{figure}[ht]\begin{center}
      \includegraphics[width=0.5\textwidth]{%
      ../output/techint-s-histogram.png}
      \caption{S: Techint - Histograma}
      \label{techint-s-histogram}
    \end{center}\end{figure}


    El protocolo que mas aparece en este escenario es \texttt{IPv4},
    tal como se observa en la figura \ref{techint-s-pie}, haciendo que
    su probabilidad sea la mayor. Esto se ve reflejado en la figura
    \ref{techint-s-histogram}, en la cual se puede ver como afecta la
    frecuencia de aparición del protocolo a la información que este
    provee.


    En este caso el nodo distinguido es el protocolo \texttt{IPv4},
    pues la información que provee es menor a la entropía de la fuente
    utilizada(1.74) , y además es el único que está por debajo de
    ella. El protocolo que mas información aporta es \texttt{LLDP}, ya
    que su frecuencia de aparición(0.15\%) es la menor en la medición
    tomada. Esto genera que las pocas veces que aparece aporte mas
    información en comparación con los protocolos que más aparecen,
    como por ejemplo
    \texttt{IPv4} cuyo porcentaje de aparición sobre el total es del 56\%
    tal como se observa en la figura \ref{techint-s-pie}.

    % ALGO MAS ACA?
    También se puede observar en la figura \ref{techint-s-pie}, la
    gran cantidad de paquetes de tipo \texttt{ARP} (12\%) que
    aparecen, en comparación con los de tipo \texttt{IPv6} (15\%) y
    \texttt{802.3} (14\%), que se encuentran en segundo lugar y tercer
    lugar respectivamente. Esto demuestra el overhead del protocolo
    \texttt{ARP} sobre el total de paquetes escuchados.

  %--------------------------------------------------------------------
  % SUBSECTION II-C: HYUNDAI
  %--------------------------------------------------------------------
  \subsection{Hyundai}

    Esta medición fue realizada en la empresa Hyundai Motor Argentina,
    en horario laboral, durante 2 horas. El edificio en donde se
    realizó la medición, cuenta con unos 30 estaciones de trabajo
    fijas (PC Desktop), y 10 Notebooks, distribuídas a través de los
    pisos del edificio. La red tiene, además, un \textbf{switch de
    nivel 2} por sector, al que se encuentran conectados los
    dispositivos que pertenecen al departamento. A su vez, cada uno de
    estos está conectado a un \textbf{switch de nivel 3} a través de
    un \textbf{enlace Gigabit punto a punto}. A su vez, hay al menos
    un \textbf{access point} en cada uno de los sectores de la
    empresa, al cual se conectan la mayoría de los dispositivos
    móviles de los empleados. También hay diversos dispositivos,
    impresoras de red, lectores de códigos de barras inalámbricos,
    cámaras IPs, teléfonos IPS, entre otros, conectados a los
    \textbf{switches} o \textbf{access point} de cada sector.

    A su vez, existen varios servidores conectados mediante un enlace
    \textbf{PPP} al switch principal, que proveen de diversos servicios a las
    estaciones de trabajo, por ejemplo: active directory, samba,
    correo electrónico, backup, acceso a bases de datos, acceso a
    sistemas, telefonía IP, acceso a internet, etc.

    La conexión fue realizada mediante una conexión cableada
    \textbf{PPP} entre el switch principal, y la computadora corriendo
    el sniffer, lo que nos permite suponer que pese a activar el modo
    promiscuo, sólo llegarán hasta la placa de red aquellos paquetes
    que se encuentren dentro de su dominio de broadcast.

    \begin{table}\begin{center}
      \begin{tabular}{|c|c|c|}
      \hline
      \textbf{Protocolo}   & \textbf{Frecuencia} & \textbf{Informacion}\\ \hline
      \texttt{ARP         }& 0.75\%     & 7.06       \\ \hline
      \texttt{IPX         }& 0.01\%     & 13.41      \\ \hline
      \texttt{IPv4        }& 98.82\%    & 0.02       \\ \hline
      \texttt{IEEE\_26734 }& 0.00\%     & 16.21      \\ \hline
      \texttt{802.3       }& 0.15\%     & 9.38       \\ \hline
      \texttt{IPv6        }& 0.27\%     & 8.55       \\ \hline
      \texttt{LLDP        }& 0.01\%     & 13.71      \\ \hline
      \end{tabular}
      \caption{S: Hyundai - Mediciones}
      \label{hyundai-s-table}
    \end{center}\end{table}

    Se pueden apreciar los resultados de las mediciones en la tabla
    \ref{hyundai-s-table}. Lo primero que se observa es una muy fuerte
    predominancia de paquetes de protocolo IPv4.

    A priori, según los resultados de esta medición, parecería que
    \texttt{ARP} no impone un overhead considerable. Sobre esto,
    decidimos comparar esta medición con el resto, y buscar en qué
    cosas se diferencian, con el fin de encontrar a qué se debe esta
    situación de alta eficiencia.

    %------------------------------------------------------------------
  	% GRAFICOS HYUNDAI - TIEMPO
    %------------------------------------------------------------------
    \begin{figure}[h]\begin{center}
      \includegraphics[width=0.5\textwidth]{%
      ../output/hyundai-s-pie.png}
      \vspace{-3em}
      \caption{S: Hyundai - Torta}
      \label{hyundai-s-pie}
    \end{center}\end{figure}

    %------------------------------------------------------------------
    % HISTOGRAMA HYUNDAI
    %------------------------------------------------------------------	
    \begin{figure}[h]\begin{center}
      \includegraphics[width=0.5\textwidth]{%
      ../output/hyundai-s-histogram.png}
      \caption{S: Hyundai - Histograma}
      \label{hyundai-s-histogram}
    \end{center}\end{figure}

    \begin{figure}[H]\begin{center}
      \includegraphics[width=0.5\textwidth]{%
      ../output/hyundai-1-s-pie.png}
      \vspace{-3em}
      \caption{S: Hyundai - Torta (primer minuto)}
      \label{hyundai-1-s-pie}
    \end{center}\end{figure}
    %
    \begin{figure}[H]\begin{center}
      \includegraphics[width=0.5\textwidth]{%
      ../output/hyundai-5-s-pie.png}
      \vspace{-3em}
      \caption{S: Hyundai - Torta (primeros 5 minutos)}
      \label{hyundai-5-s-pie}
    \end{center}\end{figure}
    %
    \begin{figure}[H]\begin{center}
      \includegraphics[width=0.5\textwidth]{%
      ../output/hyundai-10-s-pie.png}
      \vspace{-3em}
      \caption{S: Hyundai - Torta (primeros 10 minutos)}
      \label{hyundai-10-s-pie}
    \end{center}\end{figure}
    %
    \begin{figure}[H]\begin{center}
      \includegraphics[width=0.5\textwidth]{%
      ../output/hyundai-20-s-pie.png}
      \vspace{-3em}
      \caption{S: Hyundai - Torta (primeros 20 minutos)}
      \label{hyundai-20-s-pie}
    \end{center}\end{figure}
    %  
    \begin{figure}[H]\begin{center}
      \includegraphics[width=0.5\textwidth]{%
      ../output/hyundai-30-s-pie.png}
      \vspace{-3em}
      \caption{S: Hyundai - Torta (primeros 30 minutos)}
      \label{hyundai-30-s-pie}
    \end{center}\end{figure}
    %

  %
  %--------------------------------------------------------------------
  % SUBSECTION II-D: LABORATORIOS DC
  %--------------------------------------------------------------------
  %
  \subsection{Laboratorios DC}

    Esta medición fue realizada en los laboratorios del departamento de
    computación, en el horario de las 17 hs, por un lapso de 30 minutos.
    Se utilizó la herramienta explicada previamente. La red cuenta con 175
    computadoras, y se desconoce la cantidad de celulares conectados a ella.
    Los resultados fueron los siguientes.
   
    %------------------------------------------------------------------
    % TORTA LABOS-DC
    %------------------------------------------------------------------	
    \begin{figure}[ht]\begin{center}
      \includegraphics[width=0.5\textwidth]{%
      ../output/labos-dc-30m-s-pie.png}
      \vspace{-3em}
      \caption{S: Laboratorios DC - Torta}
      \label{labos-dc-30m-s-pie}
    \end{center}\end{figure}

    %------------------------------------------------------------------
    % HISTOGRAMA LABOS-DC
    %------------------------------------------------------------------	
    \begin{figure}[ht]\begin{center}
      \includegraphics[width=0.5\textwidth]{%
      ../output/labos-dc-30m-s-histogram.png}
      \caption{S: Laboratorios DC - Histograma}
      \label{labos-dc-30m-s-histogram}
    \end{center}\end{figure}


  El protocolo que más se escuchó fue \texttt{IPv4}, tal como se
  observa en la figura \ref{labos-dc-30m-s-pie}, la cantidad de
  paquetes  de este tipo fue del 82\% mientras que la de paquetes
  \texttt{ARP} fue del 11\%. Se observa que el overhead del tipo \texttt{ARP} 
  en un tiempo de 30 minutos es bajo pero no despreciable con respecto al 
  total de paquetes de la red.

  La entropía de la fuente propuesta es de 0.882, como se observa en la
  figura \ref{labos-dc-30m-s-histogram}, dejando por debajo de ella al
  protocolo \texttt{IPv4} que, al presentar una mayor probabilidad de
  aparición en la fuente, provoca que la información que aporta sea
  poca y se destaque como nodo distinguido. Podemos encontrar al protocolo 
  \texttt{802.3} que al tener poca probabilidad de aparición
  (1\%), aporta mucha información, siendo el protocolo que más información
  aporta.

    %------------------------------------------------------------------
    % SUBSECTION II-F: Conclusión
    %------------------------------------------------------------------
    \subsection{Conclusión}
    % HABLAR SOBRE EL PREDOMINIO DE IPV4 SOBRE IPV6 QUE TODAVIA NO ESTA BIEN IMPLEMENTADO EN 
    % TODOS LADOS

    % IMPACTO DE ARP: SE PIERDE MUCHO TIEMPO EN PROTOCOLOS DE "MANTENIMIENTO" QUE NO APORTAN
    % INFORMACION

    % GRAN IMPACTO DE ARP EN LAPSOS CORTOS DE ESCUCHA, EN TIEMPOS LARGOS EL OVERHEAD BAJA
    %
    %
    % También se observa el gran predominio de \texttt{IPv4} sobre \texttt{IPv6},
    % el cual actualmente no esta implementado ni funcionando en todos
    % los sistemas.


%----------------------------------------------------------------------
% SECTION III: Desarrollo - Fuente S_1
%----------+------------------------------------------------------------
\section{Desarrollo - Fuente $S_1$}
  %--------------------------------------------------------------------
  % SUBSECTION III-A: CASA
  %--------------------------------------------------------------------
  \subsection{Casa}

  Las condiciones de esta medición son las mismas que las mencionadas en la
  sección II-A, solamente se cambió la fuente utilizada y se filtraron los
  paquetes segun el tipo \texttt{ARP}.

    %------------------------------------------------------------------
    % TORTA CASA
    %------------------------------------------------------------------
    \begin{figure}[ht]\begin{center}
      \includegraphics[width=0.5\textwidth]{%
      ../output/marto-casa-3hs-s1-pie.png}
      \vspace{-2em}
      \caption{S$_1$: Casa - Torta}
      \label{marto-casa-3hs-s1-pie}
    \end{center}\end{figure}

    %------------------------------------------------------------------
    % HISTOGRAMA CASA
    %------------------------------------------------------------------
    \begin{figure}[ht]\begin{center}
      \includegraphics[width=0.5\textwidth]{%
      ../output/marto-casa-3hs-s1-histogram.png}
      \vspace{-2em}
      \caption{S$_1$: Casa - Histograma}
      \label{marto-casa-3hs-s1-histogram}
    \end{center}\end{figure}	

  La ip que más aparece en los paquetes de tipo \texttt{ARP}, dentro del
  campo \textit{destino}, es la 190.168.1.100, tal como se observa en la
  figura \ref{marto-casa-3hs-s1-pie}. Esta ip al ser la que más aparece
  dentro de las comunicaciones, será un nodo distinguido, por lo que será la
  que menos información aporte. Esto se ve reflejado en la figura
  \ref{marto-casa-3hs-s1-histogram}, en donde la información que brinda la
  ip 190.168.1.100 (representada como A) queda por debajo de la entropía de
  la fuente, la cual es 1.24.  

  Las direcciones ip que no son distinguidas tienen frecuencias y
  probabilidades de aparición muy similares, aportando así
  cantidades similares de información. Esto se puede observar en la figura
  \ref{marto-casa-3hs-s1-histogram}.


  %------------------------------------------------------------ --------
  % SUBSECTION III-B: TECHINT
  %--------------------------------------------------------------------
  \subsection{Techint}

  Las condiciones en las que se midió fueron las mismas a la detallada en la
  sección III-A, solamente se cambió la fuente utilizada y se filtraron los
  paquetes segun el tipo \texttt{ARP}.  

    %------------------------------------------------------------------
    % TORTA TECHINT
    %------------------------------------------------------------------
    \begin{figure}[H]\begin{center}
      \includegraphics[width=0.5\textwidth]{%
      ../output/techint-s1-pie.png}
      \vspace{-2em}
      \caption{S$_1$: Techint - Torta}
      \label{techint-s1-pie}
    \end{center}\end{figure}

    %------------------------------------------------------------------
    % HISTOGRAMA TECHINT
    %------------------------------------------------------------------
    \begin{figure}[H]\begin{center}
      \includegraphics[width=0.5\textwidth]{%
      ../output/techint-s1-histogram.png}
      \vspace{-2em}
      \caption{S$_1$: Techint - Histograma}
      \label{techint-s1-histogram}
    \end{center}\end{figure}
  
  En este caso se encontró un subconjunto de nodos distinguidos, los cuales
  presentan una frecuencia significativamente mayor en comparación con el
  resto de los nodos. Esto se puede observar en la figura
  \ref{techint-s1-pie}. 

  Todos estos nodos aportaron un información menor a la entropía de la
  fuente, la cual fue de 4.73. Estos estan enumerados con letras de la A a
  la J, formando así un subconjunto con 10 nodos distinguidos, tal como se
  ve en la figura \ref{techint-s1-histogram}. Dentro de estos nodos, se
  observa uno en particular nombrado como A (cuya ip es 172.23.135.254), que
  aparece con una frecuencia mayor al resto (18\%), lo que como consecuencia
  aportará menor cantidad de información.

  %--------------------------------------------------------------------
  % SUBSECTION III-C: HYUNDAI
  %--------------------------------------------------------------------
  \subsection{Hyundai}

Hay muchas variaciones de los pasajes de Lorem Ipsum disponibles, pero la mayoría sufrió alteraciones en alguna manera, ya sea porque se le agregó humor, o palabras aleatorias que no parecen ni un poco creíbles. Si vas a utilizar un pasaje de Lorem Ipsum, necesitás estar seguro de que no hay nada avergonzante escondido en el medio del texto. Todos los generadores de Lorem Ipsum que se encuentran en Internet tienden a repetir trozos predefinidos cuando sea necesario, haciendo a este el único generador verdadero (válido) en la Internet. Usa un diccionario de mas de 200 palabras provenientes del latín, combinadas con estructuras muy útiles de sentencias, para generar texto de Lorem Ipsum que parezca razonable. Este Lorem Ipsum generado siempre estará libre de repeticiones, humor agregado o palabras no características del lenguaje, etc.

    %------------------------------------------------------------------
    % TORTA TECHINT
    %------------------------------------------------------------------
    \begin{figure}[ht]\begin{center}
      \includegraphics[width=0.5\textwidth]{%
      ../output/hyundai-s1-pie.png}
      \vspace{-2em}
      \caption{S$_1$: Hyundai - Torta}
      \label{hyundai-s1-pie}
    \end{center}\end{figure}

    %------------------------------------------------------------------
    % HISTOGRAMA TECHINT
    %------------------------------------------------------------------
    \begin{figure}[ht]\begin{center}
      \includegraphics[width=0.5\textwidth]{%
      ../output/hyundai-s1-histogram.png}
      \vspace{-2em}
      \caption{S$_1$: Hyundai - Histograma}
      \label{hyundai-s1-histogram}
    \end{center}\end{figure}

  %--------------------------------------------------------------------
  % SUBSECTION III-D: LABORATORIOS DC
  %--------------------------------------------------------------------
  \subsection{Laboratorios DC}

Hay muchas variaciones de los pasajes de Lorem Ipsum disponibles, pero la mayoría sufrió alteraciones en alguna manera, ya sea porque se le agregó humor, o palabras aleatorias que no parecen ni un poco creíbles. Si vas a utilizar un pasaje de Lorem Ipsum, necesitás estar seguro de que no hay nada avergonzante escondido en el medio del texto. Todos los generadores de Lorem Ipsum que se encuentran en Internet tienden a repetir trozos predefinidos cuando sea necesario, haciendo a este el único generador verdadero (válido) en la Internet. Usa un diccionario de mas de 200 palabras provenientes del latín, combinadas con estructuras muy útiles de sentencias, para generar texto de Lorem Ipsum que parezca razonable. Este Lorem Ipsum generado siempre estará libre de repeticiones, humor agregado o palabras no características del lenguaje, etc.

    %------------------------------------------------------------------
    % TORTA LABOS-DC
    %------------------------------------------------------------------
    \begin{figure}[ht]\begin{center}
      \includegraphics[width=0.5\textwidth]{%
      ../output/labos-dc-30m-s1-pie.png}
      \vspace{-2em}
      \caption{S$_1$: Laboratorios DC - Torta}
      \label{labos-dc-30m-s1-pie}
    \end{center}\end{figure}

    %------------------------------------------------------------------
    % HISTOGRAMA LABOS-DC
    %------------------------------------------------------------------
    \begin{figure}[ht]\begin{center}
      \includegraphics[width=0.5\textwidth]{%
      ../output/labos-dc-30m-s1-histogram.png}
      %\vspace{-2em}
      \caption{S$_1$: Laboratorios DC - Histograma}
      \label{labos-dc-30m-s1-histogram}
    \end{center}\end{figure}

  %--------------------------------------------------------------------
  % SUBSECTION III-E: Gráficos
  %--------------------------------------------------------------------
  %\subsection{Gráficos}
  %\subsubsection*{Gráficos Torta}
  %\subsubsection*{Gráficos Histograma}

  %--------------------------------------------------------------------
  % SUBSECTION III-F: Conclusión
  %--------------------------------------------------------------------
  \subsection{Conclusión de las mediciones}

Lorem ipsum dolor sit amet, consectetur adipiscing elit. Vestibulum cursus ipsum sapien, a laoreet orci viverra a. Etiam tempus tempor risus non egestas. Maecenas viverra turpis at velit commodo varius. Quisque lobortis elit vitae nibh condimentum, et tincidunt velit pretium. Fusce vitae lacus mi. Suspendisse at bibendum tellus. Fusce maximus nulla libero, sed ornare ex convallis eu. In hac habitasse platea dictumst. Pellentesque commodo rhoncus turpis, sit amet laoreet ex dignissim sit amet. Maecenas in nibh ut purus consequat varius sed at nulla. Duis in magna ut augue aliquam vestibulum. Quisque rhoncus vehicula elementum. Duis tristique imperdiet molestie. Nam placerat tellus finibus laoreet ultrices. Proin feugiat aliquet convallis. Nulla tempus aliquam sem.

%----------------------------------------------------------------------
% SECTION IV: Conclusión
%----------------------------------------------------------------------
\section{Conclusión}

Fusce pellentesque egestas euismod. Lorem ipsum dolor sit amet, consectetur adipiscing elit. Maecenas eu imperdiet risus. Vestibulum ante ipsum primis in faucibus orci luctus et ultrices posuere cubilia Curae; Etiam porttitor, sapien vel vestibulum tincidunt, elit orci bibendum enim, ut dignissim massa metus in elit. Proin id imperdiet justo. Maecenas egestas, tortor quis blandit sagittis, diam ante eleifend elit, et volutpat tortor velit eget ligula. Nulla sit amet libero tellus.

Aliquam rhoncus euismod hendrerit. Integer quis nunc non enim tincidunt aliquam nec id dolor. Pellentesque et eleifend urna. Sed dictum in est eu ullamcorper. Cras suscipit scelerisque dignissim. Vestibulum at ligula orci. Phasellus sit amet mauris bibendum dolor ullamcorper fermentum id id elit. In pharetra malesuada gravida. Proin vitae ultrices tellus, at iaculis tellus. Proin suscipit quis velit sit amet finibus.

Curabitur vitae nulla libero. Suspendisse in velit id mi scelerisque iaculis ullamcorper lacinia magna. Curabitur at nulla imperdiet, vehicula dolor in, sodales erat. Pellentesque vel erat ligula. Vestibulum eu lectus ornare, gravida ex a, aliquet diam. Aliquam ut nisl sit amet dui tempus ornare. Curabitur ullamcorper, enim et laoreet finibus, enim magna molestie magna, sed semper augue odio eget neque. Sed non est posuere, malesuada magna in, luctus diam. Aliquam posuere placerat libero at sollicitudin. Sed lacus enim, maximus sit amet placerat consectetur, tincidunt eget diam.

\end{document}
