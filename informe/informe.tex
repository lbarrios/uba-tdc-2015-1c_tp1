\documentclass[final,narroweqnarray,inline]{ieee}
% In order to use the figure-defining commands in ieeefig.sty...
\usepackage{ieeefig}
% To use utf8 encoding
\usepackage[T1]{fontenc}
\usepackage[utf8]{inputenc}
\usepackage[spanish]{babel}
% to place figures
\usepackage{float}
\usepackage{graphicx}
\usepackage{geometry}
\begin{document}

%----------------------------------------------------------------------
% Title Information, Abstract and Keywords
%----------------------------------------------------------------------
\title[Trabajo Práctico N$^o$ 1: Wiretapping]{%
Trabajo Práctico N$^o$ 1: Wiretapping}

% format author this way for journal articles.
\author[SHORT NAMES]{%
	Leandro Ezequiel Barrios,
	\and
	Gonzalo Benegas,
	\and
	Martin Caravario, 
	\and
	Pedro Rodriguez 
}

% make the title
\maketitle

% do the abstract
\begin{abstract}

En el presente Trabajo Práctico utilizaremos algunas de las técnicas
provistas por la teoría de la información para estudiar y analizar
algunas redes de información. El objetivo será distinguir diversos
aspectos de la red de manera analítica. Para cumplir con nuestro
objetivo, haremos uso de dos herramientas modernas de manipulación y
análisis de paquetes: Wireshark y Scapy.

\end{abstract}

% start the main text ...

%----------------------------------------------------------------------
% SECTION I: Introduction
%----------------------------------------------------------------------
\section{ Introducción }

Construimos una herramienta que hace uso de la función
\texttt{``sniff''}, provista por la librería \textbf{Scapy} de
Python. Esta nos permitió activar el \texttt{modo promiscuo}, o
\texttt{monitor} en el caso de las placas wireless. Esto nos permitió
escuchar la red durante cierto tiempo, obteniendo todos los paquetes
que llegaban a nuestra placa de red. A partir de estos datos que
guardamos en un archivo \texttt{pcap}, se definieron dos fuentes de
información, con las cuales fuimos capaces de encontrar nodos y
protocolos distinguidos en la red. Para su visualización, elegimos la
realización de gráficos de torta e histogramas, ya que los
consideramos los más apropiados.

Para cada una de las mediciones consideramos las siguientes fuentes: 
\begin{itemize}

  \item $S = \{s_{1} \dots s_{n}\}$, provista por la cátedra, donde
  $s_{i}$ es el valor del campo \emph{type} de cada paquete de capa
  2.

  \item $S_{1} = \{s_{1} \dots s_{n}\} $, determinada por nosotros,
  donde $s_i$ es el valor del campo destino (MAC) cada paquete de
  capa 2 de tipo ARP.

\end{itemize}

\medskip

Para entender qué es lo que se obtendrá al efectuar estas mediciones,
hay que aclarar que \texttt{ARP} es un protocolo de la capa de enlace
de datos, responsable de encontrar la dirección de capa 2
(\texttt{Ethernet MAC}) que corresponde a una determinada dirección IP
(dirección de capa 3 de red).

Es decir, cada vez que un host quiere comunicarse con otro y su
dirección \texttt{MAC} no se encuentra dentro de su tabla
\texttt{ARP}, debe enviar un paquete who-has broadcast para determinar
la dirección \texttt{MAC} del host destino. De este modo, todos los
hosts del dominio de colisión de la máquina en la que se efectúa la
medición reciben dicho paquete, siendo respondido el mismo únicamente
por el host requerido, mediante un paquete \emph{reply}.

Para distinguir \emph{nodos} (símbolos) en este contexto, tomamos a
aquellos cuya probabilidad de aparición era alta, de forma tal que la
información provista por el mismo fuera menor a la entropía de la
fuente a la cuál el símbolo pertenecía. 

Tomamos esta decisión porque, según Shannon, el nivel de entropía de una
fuente habla de la máxima compresibilidad de cada bit en un mensaje enviado 
con una codificación óptima ($H(F) \leq L(C)$). Luego, si la 
fuente presenta símbolos cuya cantidad de información está
por debajo de la entropía, son símbolos que tienen mucha probabilidad de
aparecer en un mensaje en comparación con los otros, y podría ser 
conveniente representar a estos símbolos con menos bits que al resto, para
así disminuir la longitud media del código. Por ejemplo, supongamos que 
enviamos números binarios, y sabemos que el símbolo $X = "00000000"$ aparece en 
los mensajes la mitad del tiempo y que el resto de las tiras pueden aparecer 
todas con la misma
probabilidad. Entonces, $X$ brindaría una cantidad de información por debajo
de la entropía, y convendría representar a esa tira simplemente con un $0$,
y al resto de las tiras prefijarlas con el $1$. De esta manera, se ahorraría
en promedio $7$.$1/2+(-1)$.$1/2 = 3$ bits en cada mensaje.

Entonces, también analizaremos las entropías de las fuentes en cada una
de las escuchas de red realizadas, y trataremos
de concluir cuál de las dos fuentes elegidas es \emph{más compresible}.

%----------------------------------------------------------------------
% SECTION II: Desarrollo - Fuente S
%----------------------------------------------------------------------
\section{Desarrollo - Fuente S}
  %--------------------------------------------------------------------
  % SUBSECTION II-A: CASA
  %--------------------------------------------------------------------
  \subsection{Casa}

Lorem Ipsum es simplemente el texto de relleno de las imprentas y archivos de texto. Lorem Ipsum ha sido el texto de relleno estándar de las industrias desde el año 1500, cuando un impresor (N. del T. persona que se dedica a la imprenta) desconocido usó una galería de textos y los mezcló de tal manera que logró hacer un libro de textos especimen. No sólo sobrevivió 500 años, sino que tambien ingresó como texto de relleno en documentos electrónicos, quedando esencialmente igual al original. Fue popularizado en los 60s con la creación de las hojas "Letraset", las cuales contenian pasajes de Lorem Ipsum, y más recientemente con software de autoedición, como por ejemplo Aldus PageMaker, el cual incluye versiones de Lorem Ipsum.

    %------------------------------------------------------------------
    % TORTA CASA
    %------------------------------------------------------------------

   % \begin{figure}[ht]\begin{center}
   %   \includegraphics[width=0.5\textwidth]{%
   %   ../output/marto-casa-3hs-s-pie.png}
   %   \vspace{-3em}
   %   \caption{S: Casa - Torta}
   %   \label{marto-casa-3hs-s-pie}
   % \end{center}\end{figure}

    %------------------------------------------------------------------
    % HISTOGRAMA CASA
    %------------------------------------------------------------------
    %\begin{figure}[ht]\begin{center}
    %  \includegraphics[width=0.5\textwidth]{%
    %  ../output/marto-casa-3hs-s-histogram.png}
    %  \caption{S: Casa - Histograma}
    %  \label{marto-casa-3hs-s-histogram}
    %\end{center}\end{figure}

    %------------------------------------------------------------------
    % SUBSECTION II-B: TECHINT
    %------------------------------------------------------------------
    \subsection{Techint}

    Esta medición fue realizada en la empresa Techint, en el horario de
    las 11:00 am, por un lapso de 30 minutos. Se utilizó la herramienta
    desarrollada en el ejercicio anterior. Se desconoce la cantidad de
    computadoras o la topología de la red medida. Los resultados fueron
    los siguientes.
    
    %------------------------------------------------------------------
    % TORTA TECHINT
    %------------------------------------------------------------------
    
    \begin{figure}[H]\begin{center}
      \includegraphics[width=0.5\textwidth]{%
      ../output/techint-s-pie.png}
      \vspace{-3em}
      \caption{S: Techint - Torta}
      \label{techint-s-pie}
    \end{center}\end{figure}
 
    %------------------------------------------------------------------
    % HISTOGRAMA TECHINT
    %------------------------------------------------------------------
     
    \begin{figure}[H]\begin{center}
      \includegraphics[width=0.5\textwidth]{%
      ../output/techint-s-histogram.png}
      \caption{S: Techint - Histograma}
      \label{techint-s-histogram}
    \end{center}\end{figure}


    Como se observa en la figura 1, el protocolo que mas aparece en
    este escenario es \texttt{IPv4}, haciendo que su probabilidad sea 
    la mayor. Esto se ve reflejado en la figura 2, en la cual 
    se puede ver como afecta la frecuencia de aparición del protocolo a 
    la información que este provee.

    % CHEQUEAR ESTE PARRAFO, QUIZAS ES HUMO
    También se observa el gran predominio de \texttt{IPv4} sobre \texttt{IPv6},
    el cual actualmente no esta implementado ni funcionando unicamente en todos
    los sistemas.

    En este caso el nodo distinguido es el protocolo \texttt{IPv4}, 
    pues la información que provee es menor a la entropía de la fuente 
    utilizada, y además es el único que está por debajo de ella. El
    protocolo que mas información aporta es \texttt{LLDP}, ya que
    su frecuencia de aparición es la menor en la medición tomada. Esto
    genera que las pocas veces que aparece aporte mas información en 
    comparación con los protocolos que mas aparecen (\texttt{IPv4}),
    tal como se observa en la figura 2.

    % ALGO MAS ACA?
    También se puede observar en la figura 1, la gran cantidad de paquetes 
    de tipo \texttt{ARP} (12\%) que aparecen, en comparación con los
    de tipo \texttt{IPv6} (15\%) y \texttt{802.3} (14\%), que se encuentran
    en segundo lugar y tercer lugar respectivamente. Esto demuestra el overhead
    del protocolo \texttt{ARP} sobre el total de paquetes escuchados.

  %--------------------------------------------------------------------
  % SUBSECTION II-C: HYUNDAI
  %--------------------------------------------------------------------
  \subsection{Hyundai}

Lorem Ipsum es simplemente el texto de relleno de las imprentas y archivos de texto. Lorem Ipsum ha sido el texto de relleno estándar de las industrias desde el año 1500, cuando un impresor (N. del T. persona que se dedica a la imprenta) desconocido usó una galería de textos y los mezcló de tal manera que logró hacer un libro de textos especimen. No sólo sobrevivió 500 años, sino que tambien ingresó como texto de relleno en documentos electrónicos, quedando esencialmente igual al original. Fue popularizado en los 60s con la creación de las hojas "Letraset", las cuales contenian pasajes de Lorem Ipsum, y más recientemente con software de autoedición, como por ejemplo Aldus PageMaker, el cual incluye versiones de Lorem Ipsum.

    %------------------------------------------------------------------
  	% GRAFICOS HYUNDAI - TIEMPO
    %------------------------------------------------------------------
    \begin{figure}[H]\begin{center}
      \includegraphics[width=0.5\textwidth]{%
      ../output/hyundai-1-s-pie.png}
      \vspace{-3em}
      \caption{S: Hyundai - Torta (1 minuto)}
      \label{hyundai-1-s-pie}
    \end{center}\end{figure}
    %
    \begin{figure}[H]\begin{center}
      \includegraphics[width=0.5\textwidth]{%
      ../output/hyundai-5-s-pie.png}
      \vspace{-3em}
      \caption{S: Hyundai - Torta (5 minutos)}
      \label{hyundai-5-s-pie}
    \end{center}\end{figure}
    %
    \begin{figure}[H]\begin{center}
      \includegraphics[width=0.5\textwidth]{%
      ../output/hyundai-10-s-pie.png}
      \vspace{-3em}
      \caption{S: Hyundai - Torta (10 minutos)}
      \label{hyundai-10-s-pie}
    \end{center}\end{figure}
    %
    \begin{figure}[H]\begin{center}
      \includegraphics[width=0.5\textwidth]{%
      ../output/hyundai-20-s-pie.png}
      \vspace{-3em}
      \caption{S: Hyundai - Torta (20 minutos)}
      \label{hyundai-20-s-pie}
    \end{center}\end{figure}
    %  
    \begin{figure}[H]\begin{center}
      \includegraphics[width=0.5\textwidth]{%
      ../output/hyundai-30-s-pie.png}
      \vspace{-3em}
      \caption{S: Hyundai - Torta (30 minutos)}
      \label{hyundai-30-s-pie}
    \end{center}\end{figure}
    %
    \begin{figure}[H]\begin{center}
      \includegraphics[width=0.5\textwidth]{%
      ../output/hyundai-s-pie.png}
      \vspace{-3em}
      \caption{S: Hyundai - Torta (Total)}
      \label{hyundai-s-pie}
    \end{center}\end{figure}

    %------------------------------------------------------------------
    % HISTOGRAMA HYUNDAI
    %------------------------------------------------------------------	
    \begin{figure}[ht]\begin{center}
      \includegraphics[width=0.5\textwidth]{%
      ../output/hyundai-s-histogram.png}
      \caption{S: Hyundai - Histograma}
      \label{hyundai-s-histogram}
    \end{center}\end{figure}

  %
  %--------------------------------------------------------------------
  % SUBSECTION II-D: LABORATORIOS DC
  %--------------------------------------------------------------------
  %
  \subsection{Laboratorios DC}

Lorem Ipsum es simplemente el texto de relleno de las imprentas y archivos de texto. Lorem Ipsum ha sido el texto de relleno estándar de las industrias desde el año 1500, cuando un impresor (N. del T. persona que se dedica a la imprenta) desconocido usó una galería de textos y los mezcló de tal manera que logró hacer un libro de textos especimen. No sólo sobrevivió 500 años, sino que tambien ingresó como texto de relleno en documentos electrónicos, quedando esencialmente igual al original. Fue popularizado en los 60s con la creación de las hojas "Letraset", las cuales contenian pasajes de Lorem Ipsum, y más recientemente con software de autoedición, como por ejemplo Aldus PageMaker, el cual incluye versiones de Lorem Ipsum.

    %------------------------------------------------------------------
    % TORTA LABOS-DC
    %------------------------------------------------------------------	
    \begin{figure}[ht]\begin{center}
      \includegraphics[width=0.5\textwidth]{%
      ../output/labos-dc-30m-s-pie.png}
      \vspace{-3em}
      \caption{S: Laboratorios DC - Torta}
      \label{labos-dc-30m-s-pie}
    \end{center}\end{figure}

    %------------------------------------------------------------------
    % HISTOGRAMA LABOS-DC
    %------------------------------------------------------------------	
    \begin{figure}[ht]\begin{center}
      \includegraphics[width=0.5\textwidth]{%
      ../output/labos-dc-30m-s-histogram.png}
      \caption{S: Laboratorios DC - Histograma}
      \label{labos-dc-30m-s-histogram}
    \end{center}\end{figure}

    %------------------------------------------------------------------
    % SUBSECTION II-E: Gráficos
    %------------------------------------------------------------------
    %\subsection{Gráficos}
    %\subsubsection*{Gráficos Torta}
    %\subsubsection*{Gráficos Histograma}

    %------------------------------------------------------------------
    % SUBSECTION II-F: Conclusión
    %------------------------------------------------------------------
    \subsection{Conclusión}
    % HABLAR SOBRE EL PREDOMINIO DE IPV4 SOBRE IPV6 QUE TODAVIA NO ESTA BIEN IMPLEMENTADO EN 
    % TODOS LADOS

    % IMPACTO DE ARP: SE PIERDE MUCHO TIEMPO EN PROTOCOLOS DE "MANTENIMIENTO" QUE NO APORTAN
    % INFORMACION

    % GRAN IMPACTO DE ARP EN LAPSOS CORTOS DE ESCUCHA, EN TIEMPOS LARGOS EL OVERHEAD BAJA
Al contrario del pensamiento popular, el texto de Lorem Ipsum no es simplemente texto aleatorio. Tiene sus raices en una pieza clásica de la literatura del Latin, que data del año 45 antes de Cristo, haciendo que este adquiera mas de 2000 años de antiguedad. Richard McClintock, un profesor de Latin de la Universidad de Hampden-Sydney en Virginia, encontró una de las palabras más oscuras de la lengua del latín, ``consecteur'', en un pasaje de Lorem Ipsum, y al seguir leyendo distintos textos del latín, descubrió la fuente indudable. Lorem Ipsum viene de las secciones 1.10.32 y 1.10.33 de ``de Finnibus Bonorum et Malorum'' (Los Extremos del Bien y El Mal) por Cicero, escrito en el año 45 antes de Cristo. Este libro es un tratado de teoría de éticas, muy popular durante el Renacimiento. La primera linea del Lorem Ipsum, ``Lorem ipsum dolor sit amet..'', viene de una linea en la sección 1.10.32

El trozo de texto estándar de Lorem Ipsum usado desde el año 1500 es reproducido debajo para aquellos interesados. Las secciones 1.10.32 y 1.10.33 de ``de Finibus Bonorum et Malorum'' por Cicero son también reproducidas en su forma original exacta, acompañadas por versiones en Inglés de la traducción realizada en 1914 por H. Rackham.

%----------------------------------------------------------------------
% SECTION III: Desarrollo - Fuente S_1
%----------+------------------------------------------------------------
\clearpage
\section{Desarrollo - Fuente $S_1$}
  %--------------------------------------------------------------------
  % SUBSECTION II-A: CASA
  %--------------------------------------------------------------------
  \subsection{Casa}

Hay muchas variaciones de los pasajes de Lorem Ipsum disponibles, pero la mayoría sufrió alteraciones en alguna manera, ya sea porque se le agregó humor, o palabras aleatorias que no parecen ni un poco creíbles. Si vas a utilizar un pasaje de Lorem Ipsum, necesitás estar seguro de que no hay nada avergonzante escondido en el medio del texto. Todos los generadores de Lorem Ipsum que se encuentran en Internet tienden a repetir trozos predefinidos cuando sea necesario, haciendo a este el único generador verdadero (válido) en la Internet. Usa un diccionario de mas de 200 palabras provenientes del latín, combinadas con estructuras muy útiles de sentencias, para generar texto de Lorem Ipsum que parezca razonable. Este Lorem Ipsum generado siempre estará libre de repeticiones, humor agregado o palabras no características del lenguaje, etc.

    %------------------------------------------------------------------
    % TORTA CASA
    %------------------------------------------------------------------
    \begin{figure}[ht]\begin{center}
      \includegraphics[width=0.5\textwidth]{%
      ../output/marto-casa-3hs-s1-pie.png}
      \vspace{-2em}
      \caption{S$_1$: Casa - Torta}
      \label{marto-casa-3hs-s1-pie}
    \end{center}\end{figure}

    %------------------------------------------------------------------
    % HISTOGRAMA CASA
    %------------------------------------------------------------------
    \begin{figure}[ht]\begin{center}
      \includegraphics[width=0.5\textwidth]{%
      ../output/marto-casa-3hs-s1-histogram.png}
      \vspace{-2em}
      \caption{S$_1$: Casa - Histograma}
      \label{marto-casa-3hs-s1-histogram}
    \end{center}\end{figure}	

  %--------------------------------------------------------------------
  % SUBSECTION II-B: TECHINT
  %--------------------------------------------------------------------
  \subsection{Techint}

Hay muchas variaciones de los pasajes de Lorem Ipsum disponibles, pero la mayoría sufrió alteraciones en alguna manera, ya sea porque se le agregó humor, o palabras aleatorias que no parecen ni un poco creíbles. Si vas a utilizar un pasaje de Lorem Ipsum, necesitás estar seguro de que no hay nada avergonzante escondido en el medio del texto. Todos los generadores de Lorem Ipsum que se encuentran en Internet tienden a repetir trozos predefinidos cuando sea necesario, haciendo a este el único generador verdadero (válido) en la Internet. Usa un diccionario de mas de 200 palabras provenientes del latín, combinadas con estructuras muy útiles de sentencias, para generar texto de Lorem Ipsum que parezca razonable. Este Lorem Ipsum generado siempre estará libre de repeticiones, humor agregado o palabras no características del lenguaje, etc.

    %------------------------------------------------------------------
    % TORTA TECHINT
    %------------------------------------------------------------------
    \begin{figure}[ht]\begin{center}
      \includegraphics[width=0.5\textwidth]{%
      ../output/techint-s1-pie.png}
      \vspace{-2em}
      \caption{S$_1$: Techint - Torta}
      \label{techint-s1-pie}
    \end{center}\end{figure}

    %------------------------------------------------------------------
    % HISTOGRAMA TECHINT
    %------------------------------------------------------------------
    \begin{figure}[ht]\begin{center}
      \includegraphics[width=0.5\textwidth]{%
      ../output/techint-s1-histogram.png}
      \vspace{-2em}
      \caption{S$_1$: Techint - Histograma}
      \label{techint-s1-histogram}
    \end{center}\end{figure}


  %--------------------------------------------------------------------
  % SUBSECTION II-C: HYUNDAI
  %--------------------------------------------------------------------
  \subsection{Hyundai}

Hay muchas variaciones de los pasajes de Lorem Ipsum disponibles, pero la mayoría sufrió alteraciones en alguna manera, ya sea porque se le agregó humor, o palabras aleatorias que no parecen ni un poco creíbles. Si vas a utilizar un pasaje de Lorem Ipsum, necesitás estar seguro de que no hay nada avergonzante escondido en el medio del texto. Todos los generadores de Lorem Ipsum que se encuentran en Internet tienden a repetir trozos predefinidos cuando sea necesario, haciendo a este el único generador verdadero (válido) en la Internet. Usa un diccionario de mas de 200 palabras provenientes del latín, combinadas con estructuras muy útiles de sentencias, para generar texto de Lorem Ipsum que parezca razonable. Este Lorem Ipsum generado siempre estará libre de repeticiones, humor agregado o palabras no características del lenguaje, etc.

    %------------------------------------------------------------------
    % TORTA TECHINT
    %------------------------------------------------------------------
    \begin{figure}[ht]\begin{center}
      \includegraphics[width=0.5\textwidth]{%
      ../output/hyundai-s1-pie.png}
      \vspace{-2em}
      \caption{S$_1$: Hyundai - Torta}
      \label{hyundai-s1-pie}
    \end{center}\end{figure}

    %------------------------------------------------------------------
    % HISTOGRAMA TECHINT
    %------------------------------------------------------------------
    \begin{figure}[ht]\begin{center}
      \includegraphics[width=0.5\textwidth]{%
      ../output/hyundai-s1-histogram.png}
      \vspace{-2em}
      \caption{S$_1$: Hyundai - Histograma}
      \label{hyundai-s1-histogram}
    \end{center}\end{figure}

  %--------------------------------------------------------------------
  % SUBSECTION II-D: LABORATORIOS DC
  %--------------------------------------------------------------------
  \subsection{Laboratorios DC}

Hay muchas variaciones de los pasajes de Lorem Ipsum disponibles, pero la mayoría sufrió alteraciones en alguna manera, ya sea porque se le agregó humor, o palabras aleatorias que no parecen ni un poco creíbles. Si vas a utilizar un pasaje de Lorem Ipsum, necesitás estar seguro de que no hay nada avergonzante escondido en el medio del texto. Todos los generadores de Lorem Ipsum que se encuentran en Internet tienden a repetir trozos predefinidos cuando sea necesario, haciendo a este el único generador verdadero (válido) en la Internet. Usa un diccionario de mas de 200 palabras provenientes del latín, combinadas con estructuras muy útiles de sentencias, para generar texto de Lorem Ipsum que parezca razonable. Este Lorem Ipsum generado siempre estará libre de repeticiones, humor agregado o palabras no características del lenguaje, etc.

    %------------------------------------------------------------------
    % TORTA LABOS-DC
    %------------------------------------------------------------------
    \begin{figure}[ht]\begin{center}
      \includegraphics[width=0.5\textwidth]{%
      ../output/labos-dc-30m-s1-pie.png}
      \vspace{-2em}
      \caption{S$_1$: Laboratorios DC - Torta}
      \label{labos-dc-30m-s1-pie}
    \end{center}\end{figure}

    %------------------------------------------------------------------
    % HISTOGRAMA LABOS-DC
    %------------------------------------------------------------------
    \begin{figure}[ht]\begin{center}
      \includegraphics[width=0.5\textwidth]{%
      ../output/labos-dc-30m-s1-histogram.png}
      %\vspace{-2em}
      \caption{S$_1$: Laboratorios DC - Histograma}
      \label{labos-dc-30m-s1-histogram}
    \end{center}\end{figure}

  %--------------------------------------------------------------------
  % SUBSECTION II-E: Gráficos
  %--------------------------------------------------------------------
  %\subsection{Gráficos}
  %\subsubsection*{Gráficos Torta}
  %\subsubsection*{Gráficos Histograma}

  %--------------------------------------------------------------------
  % SUBSECTION II-F: Conclusión
  %--------------------------------------------------------------------
  \subsection{Conclusión de las mediciones}

Lorem ipsum dolor sit amet, consectetur adipiscing elit. Vestibulum cursus ipsum sapien, a laoreet orci viverra a. Etiam tempus tempor risus non egestas. Maecenas viverra turpis at velit commodo varius. Quisque lobortis elit vitae nibh condimentum, et tincidunt velit pretium. Fusce vitae lacus mi. Suspendisse at bibendum tellus. Fusce maximus nulla libero, sed ornare ex convallis eu. In hac habitasse platea dictumst. Pellentesque commodo rhoncus turpis, sit amet laoreet ex dignissim sit amet. Maecenas in nibh ut purus consequat varius sed at nulla. Duis in magna ut augue aliquam vestibulum. Quisque rhoncus vehicula elementum. Duis tristique imperdiet molestie. Nam placerat tellus finibus laoreet ultrices. Proin feugiat aliquet convallis. Nulla tempus aliquam sem.

%----------------------------------------------------------------------
% SECTION IV: Conclusión
%----------------------------------------------------------------------
\section{Conclusión}

Fusce pellentesque egestas euismod. Lorem ipsum dolor sit amet, consectetur adipiscing elit. Maecenas eu imperdiet risus. Vestibulum ante ipsum primis in faucibus orci luctus et ultrices posuere cubilia Curae; Etiam porttitor, sapien vel vestibulum tincidunt, elit orci bibendum enim, ut dignissim massa metus in elit. Proin id imperdiet justo. Maecenas egestas, tortor quis blandit sagittis, diam ante eleifend elit, et volutpat tortor velit eget ligula. Nulla sit amet libero tellus.

Aliquam rhoncus euismod hendrerit. Integer quis nunc non enim tincidunt aliquam nec id dolor. Pellentesque et eleifend urna. Sed dictum in est eu ullamcorper. Cras suscipit scelerisque dignissim. Vestibulum at ligula orci. Phasellus sit amet mauris bibendum dolor ullamcorper fermentum id id elit. In pharetra malesuada gravida. Proin vitae ultrices tellus, at iaculis tellus. Proin suscipit quis velit sit amet finibus.

Curabitur vitae nulla libero. Suspendisse in velit id mi scelerisque iaculis ullamcorper lacinia magna. Curabitur at nulla imperdiet, vehicula dolor in, sodales erat. Pellentesque vel erat ligula. Vestibulum eu lectus ornare, gravida ex a, aliquet diam. Aliquam ut nisl sit amet dui tempus ornare. Curabitur ullamcorper, enim et laoreet finibus, enim magna molestie magna, sed semper augue odio eget neque. Sed non est posuere, malesuada magna in, luctus diam. Aliquam posuere placerat libero at sollicitudin. Sed lacus enim, maximus sit amet placerat consectetur, tincidunt eget diam.

\end{document}
