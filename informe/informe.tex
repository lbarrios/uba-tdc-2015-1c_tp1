\documentclass[final,narroweqnarray,inline]{ieee}
% In order to use the figure-defining commands in ieeefig.sty...
\usepackage{ieeefig}
% To use utf8 encoding
\usepackage[utf8]{inputenc}
\begin{document}

%----------------------------------------------------------------------
% Title Information, Abstract and Keywords
%----------------------------------------------------------------------
\title[Trabajo Practico Nº 1: Wiretapping]{%
       Trabajo Practico Nº 1: Wiretapping}

% format author this way for journal articles.
\author[SHORT NAMES]{%
	Leandro Ezequiel Barrios,
	\and
	Gonzalo Benegas,
	\and
	Martin Caravario, 
	\and
	Pedro Rodriguez 
}

% make the title
\maketitle               

% do the abstract
\begin{abstract}
En el presente Trabajo Práctico utilizaremos algunas de las técnicas
provistas por la teoría de la información para estudiar y analizar algunas
redes de información. El objetivo será distinguir diversos aspectos de la
red de manera analítica. Para cumplir con nuestro objetivo, haremos uso de
dos herramientas modernas de manipulación y análisis de paquetes: Wireshark
y Scapy.
\end{abstract}

% start the main text ...


%
% Introducción
% Desarrollo Fuente S
%   Medicion Techint | Casa | Hyundai | Labos | Conclusión
% Desarrollo Fuente S_1
%   Medicion Techint | Casa | Hyundai | Labos | Conclusión
% Conclusión
%----------------------------------------------------------------------
% SECTION I: Introduction
%----------------------------------------------------------------------
\section{ Introducción }

Construimos una herramienta que hace uso de la
función ``sniff'', provista por la librería Scapy de Python. Esta nos permitió escuchar durante
cierto tiempo la red local y guardarnos todos los paquetes que llegaban a
nuestra placa de red y eran levantados por esta. A partir de estos datos
que guardamos, fuimos capaces de encontrar  nodos y protocolos
distinguidos en la red a partir de la realización de gráficos de torta e histogramas apropiados.

Para cada una de las mediciones consideramos las siguientes fuentes: 
\begin{enumerate}
  \item $S = \{s_{1} \dots s_{n}\}$, provista por la cátedra, donde $s_{i}$ es el valor del campo
        \emph{type} del frame de capa 2. 
  \item $S_{1} = \{s_{1} \dots s_{n}\} $, elegida por nosotros, con todos los paquetes escuchados 
        siendo de tipo ARP.
\end{enumerate}

\medskip
Para entender qué es lo que se obtendrá al efectuar estas mediciones, hay que aclarar que
ARP es un protocolo de la capa de enlace de datos, responsable de encontrar
la dirección de capa 2 (Ethernet MAC) que corresponde a una determinada
dirección IP (dirección de capa 3 de enlace). Es decir, cada vez que un host
quiere comunicarse con otro y su dirección MAC no se encuentra dentro de su
tabla ARP, debe enviar un paquete who-has broadcast para determinar la dirección MAC
del host destino. De este modo, todos los hosts del dominio de colisión de
la máquina en la que se efectúa la medición reciben dicho paquete, siendo respondido el mismo únicamente
por el host requerido, mediante un paquete \emph{rep}, mediante un paquete \emph{reply}.
Para distinguir nodos (símbolos) en este contexto, tomamos a aquellos cuya
probabilidad de aparición era alta, de forma tal que la información provista
por el mismo fuera menor a la entropía de la fuente a la cuál el símbolo pertenecía.

\newpage
%----------------------------------------------------------------------
% SECTION II: Desarrollo - Fuente S
%----------------------------------------------------------------------
\section{Desarrollo - Fuente S}
  %--------------------------------------------------------------------
  % SUBSECTION II-A: CASA
  %--------------------------------------------------------------------
  \subsection{Casa}

  %--------------------------------------------------------------------
  % SUBSECTION II-B: TECHINT
  %--------------------------------------------------------------------
  \subsection{Techint}

  %--------------------------------------------------------------------
  % SUBSECTION II-C: HYUNDAI
  %--------------------------------------------------------------------
  \subsection{Hyundai}

  %--------------------------------------------------------------------
  % SUBSECTION II-D: LABORATORIOS DC
  %--------------------------------------------------------------------
  \subsection{Laboratorios DC}

  %--------------------------------------------------------------------
  % SUBSECTION II-E: Gráficos
  %--------------------------------------------------------------------
  \subsection{Gráficos}
  \subsubsection*{Gráficos Torta}
  \subsubsection*{Gráficos Histograma}

  %--------------------------------------------------------------------
  % SUBSECTION II-F: Conclusión
  %--------------------------------------------------------------------
  \subsection{Conclusión}

\newpage
%----------------------------------------------------------------------
% SECTION III: Desarrollo - Fuente S
%----------------------------------------------------------------------
\section{Desarrollo - Fuente $S_1$}
  %--------------------------------------------------------------------
  % SUBSECTION II-A: CASA
  %--------------------------------------------------------------------
  \subsection{Casa}

  %--------------------------------------------------------------------
  % SUBSECTION II-B: TECHINT
  %--------------------------------------------------------------------
  \subsection{Techint}

  %--------------------------------------------------------------------
  % SUBSECTION II-C: HYUNDAI
  %--------------------------------------------------------------------
  \subsection{Hyundai}

  %--------------------------------------------------------------------
  % SUBSECTION II-D: LABORATORIOS DC
  %--------------------------------------------------------------------
  \subsection{Laboratorios DC}

  %--------------------------------------------------------------------
  % SUBSECTION II-E: Gráficos
  %--------------------------------------------------------------------
  \subsection{Gráficos}
  \subsubsection*{Gráficos Torta}
  \subsubsection*{Gráficos Histograma}

  %--------------------------------------------------------------------
  % SUBSECTION II-F: Conclusión
  %--------------------------------------------------------------------
  \subsection{Conclusión}

\newpage
%----------------------------------------------------------------------
% SECTION IV: Conclusión
%----------------------------------------------------------------------
\section{Conclusión}

% do the biliography:
\bibliographystyle{IEEEbib}
\bibliography{my-bibliography-file}

% where ``my-bibliography-file.bib'' is the name of the file with all the 
% BibTeX entries.

% do the biographies...
%\begin{biography}{Gregory L. Plett}
%  A bio with no face...
%\end{biography}

% If you want a picture with your biography, then specify the name of
% the postscript file in square brackets. That is, uncomment the
% following three lines and change the name of "face.ps" to the name of 
% your file.
%\begin{biography}[face.ps]{Gregory L. Plett}
%  A bio with a face...
%\end{biography}

\end{document}
